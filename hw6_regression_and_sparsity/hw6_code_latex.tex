\documentclass[10pt]{article}

\usepackage{graphicx}
\usepackage{amsmath,amsfonts,amssymb}
\usepackage{listings} % for code
\usepackage{color} % for code
\usepackage{xcolor} % for code
\usepackage{hyperref}  % for urls and hyperlinks

\setlength{\textwidth}{6.2in}
\setlength{\oddsidemargin}{0.3in}
\setlength{\evensidemargin}{0in}
\setlength{\textheight}{8.9in}
\setlength{\voffset}{-1in}
\setlength{\headsep}{26pt}
\setlength{\parindent}{0pt}
\setlength{\parskip}{5pt}

% TODO: put this in a python style macros file!
% set up python text formatting
% thanks to ptps://stackoverflow.com/questions/3175105/inserting-code-in-this-latex-document-with-indentation
% and ptps://tex.stackexchange.com/questions/172048/how-can-i-define-a-custom-class-of-keywords-in-listings
\definecolor{gray}{rgb}{0.5,0.5,0.5}
\definecolor{ipython_red}{RGB}{186, 33, 33}
\definecolor{ipython_green}{RGB}{0, 128, 0}
\definecolor{ipython_cyan}{RGB}{64, 128, 128}
\definecolor{ipython_purple}{RGB}{170, 34, 255}
\makeatletter
%\lst@InstallKeywords k{attributes}{attributestyle}\slshape{attributestyle}{}ld
\makeatother\lstset{frame=tb,
  language=Python,
  aboveskip=3mm,
  belowskip=3mm,
  showstringspaces=false,
  columns=flexible,
  basicstyle={\small\ttfamily},
  numbers=none,
  numberstyle=\tiny\color{gray},
  keywordstyle=\color{ipython_green},
  commentstyle=\color{ipython_cyan},
  %moreattributes={Hi, +, -, /, <, >, =, *, \%}, 
  %attributestyle = \bfseries\color{ipython_purple}, 
  stringstyle=\color{ipython_red},
  breaklines=true,
  breakatwhitespace=true,
  tabsize=3
}

% define a command for the norm
% thanks to ptps://tex.stackexchange.com/questions/107186/how-to-write-norm-which-adjusts-its-size
\newcommand{\norm}[1]{\left\lVert#1\right\rVert}


\input{../../../macros.tex}  % input some useful macros

\begin{document}

%--------------------------------------------------------------------------
% CODE:
\centerline{\Large{Python Code}}
\centerline{(\small{\textit{hw6\_midterm2.py} in the GitHub repository linked at the beginning of this document)}}
\vskip 10pt
\hrule


\begin{lstlisting}     
"""
hw6_midterm2.py

Python code for Homework 6 (Midterm #2),
AMATH 584, Fall 2020

Author: Jacqueline Nugent 
Last Modified: December 14, 2020
"""
import random
import numpy as np
import matplotlib.pyplot as plt
import matplotlib.patches as mpatches

from mnist import MNIST
from sklearn import linear_model
from tabulate import tabulate


###############################################
###########     Set file paths     ############
###############################################
# data available from http://yann.lecun.com/exdb/mnist/
file_dir = '/Users/jmnugent/Documents/__Year_3_2020-2021/AMATH_584-Numerical_Linear_Algebra/Homework/python/amath584/hw6_regression_and_sparsity/'
data_dir = file_dir + 'data/'
save_dir = file_dir + 'figures/'


###############################################
##### Some useful functions - used later ######
###############################################

def convert_label(label):
    """Convert an integer label (0-9) into a vector.
    """
    vector = np.zeros(10)
    vector[label-1] = 1

    return vector


def reshape_pic(array, n=28):
    """Reshaped input array into nxn array
    """
    new = np.reshape(array, (n, n))
    
    return new


def set_label(vec):
    """
    Assigns the 0-9 label to the label vector based on which
    entry has the largest magnitude.
    """
    ind = np.where(np.abs(vec) == np.max(np.abs(vec)))[0][0]
    if ind == 9:
        label = 0
    else:
        label = ind+1
    
    return label


def check_accuracy(labels, truth):
    """
    Returns the number of correct and incorrect labels.
        labels = 2D matrix of size (10, #samples) of the modeled labels
        truth = 1D list of the actual labels (0-9)
    Ex: [n_right, n_wrong] = check_accuracy(lasso_labs, test_labs)
    """
    n_right = 0
    n_wrong = 0
    
    for i in range(labels.shape[1]):
        error = set_label(labels[:, i]) - truth[i]
        if error == 0:
            n_right += 1
        else:
            n_wrong += 1
            
    return [n_right, n_wrong]


def check_accuracy_all(labels, truth):
    """
    Returns the number of correct and incorrect labels (with 1 loading matrix)
        labels = 1D matrix of size (#samples) of the modeled labels
        truth = 1D list of the actual labels (0-9)
    Ex: [n_right, n_wrong] = check_accuracy(lasso_labs, test_labs)
    """
    n_right = 0
    n_wrong = 0
    
    labels_rounded = [int(np.round(x)) for x in labels]
    for i in range(len(labels_rounded)):
        if labels_rounded[i] == truth[i]:
            n_right += 1
        else:
            n_wrong += 1
            
    return [n_right, n_wrong]


def plot_loadings_bar(A, method, name, n=28, save=False,
                      save_dir=save_dir):
    """
    Bar plot of the loadings for each digit
    
        A = matrix of weights (10xn^2)
        method = method name (e.g. 'Lasso (alpha=0.1)')
        name = name for the file to save (e.g. 'lasso_01')
        n = dimension of picture in each direction (default=28 pixels)
        save = True if you want to save the image
        save_dir = directory to save the file to
    """
    if A.shape != (10, n**2):
        raise Exception('Weight matrix A has incorrect dimensions: {}'.format(A.shape))
    
    fig, axes = plt.subplots(2, 5, figsize=(18, 5))
    plt.subplots_adjust(hspace=0.5)
    
    for r in range(2):
        for c in range(5):
            ax = axes[r, c]
            
            if r == 0:
                i = c
            else:
                i = c+5
        
            ax.bar(np.arange(0, n**2), A[i, :].flatten())
            ax.axhline(0, linewidth=1, alpha=0.25, color='k')

            if i == 9:
                ax.set_title('0')
            else:
                ax.set_title(str(i+1))

            ax.get_xaxis().set_ticks([])
            ax.get_yaxis().set_ticks([])

    plt.suptitle(method, fontsize=18)
    
    if save:
        plt.savefig(save_dir + 'bar_plot_loadings_' + name + '.png', dpi=300,
                    bbox_inches='tight')
    
    plt.show()


def plot_weights(A, method, name, n=28, save=False, no_zero=False, save_dir=save_dir):
    """
    Plot (and save) the weights for each method.
    
        A = matrix of weights (10xn^2)
        method = method name (e.g. 'Lasso (alpha=0.1)')
        name = name for the file to save (e.g. 'lasso_01')
        n = dimension of picture in each direction (default=28 pixels)
        save = True if you want to save the image
        no_zero = plot by omitting zeros (to see sparsity)
        save_dir = directory to save the file to
    """
    if A.shape != (10, n**2):
        raise Exception('Weight matrix A has incorrect dimensions: {}'.format(A.shape))
    
    if no_zero:
        A = np.where(A==0, np.nan, A)
        name += '_no_zeros'
    
    fig, axes = plt.subplots(2, 5, figsize=(12, 4))

    for r in range(2):
        for c in range(5):
            ax = axes[r, c]
            
            if r == 0:
                i = c
            else:
                i = c+5
        
            ax.set_aspect('equal')
            ax.pcolormesh(reshape_pic(A[i, :], n=n), cmap='gray')
        
            if i == 9:
                ax.set_title('0')
            else:
                ax.set_title(str(i+1))

            ax.get_xaxis().set_ticks([])
            ax.get_yaxis().set_ticks([])

    plt.suptitle(method)
    
    if save:
        plt.savefig(save_dir + 'weight_matrix_' + name + '.png', dpi=300,
                    bbox_inches='tight')
    plt.show()

def plot_hist_nonzero(A, method, name, save=False, save_dir=save_dir,
                      xlim=(-.02, .02)):
    """
    Plot (and save) a histogram of the aboslute value of nonzero points of the matrix.
    
        A = matrix of weights (10xn^2)
        method = method name (e.g. 'Lasso (alpha=0.1)')
        name = name for the file to save (e.g. 'lasso_01')
        save = True if you want to save the image
        save_dir = directory to save the file to
        xlim = limits on x axis (default=(-.02, .02), for pinv)
    """
    fig, axes = plt.subplots(2, 5, figsize=(12, 4))
    plt.subplots_adjust(hspace=0.6, wspace=0.4)
    
    for r in range(2):
        for c in range(5):
            ax = axes[r, c]
            
            if r == 0:
                i = c
            else:
                i = c+5
                
            flat = A[i, :].flatten()
            nonzero = flat[flat != 0.0]
            ax.hist(nonzero, bins=100)
            
            ax.set_yscale('log')
            ax.set_ylim((1, 1000))
            ax.set_xlim(xlim)
        
            if i == 9:
                ax.set_title('0')
            else:
                ax.set_title(str(i+1))

    plt.suptitle('Histogram of nonzero loadings ({})'.format(method))
    
    if save:
        plt.savefig(save_dir + 'nonzero_hist_' + name + '.png', dpi=300,
                    bbox_inches='tight')

    plt.show()    


def check_accuracy_digit(labels, truth):
    """
    Returns the number of correct and incorrect labels. # wrong is the
    number of times the model identifies the digit when it's not, not
    (len(truth) - n_right).
        labels = 2D matrix of size (10, #samples) of the modeled labels
        truth = 1D list of the actual labels (0-9)
    Ex: [n_right, n_wrong] = check_accuracy(lasso_digit_labels, test_labs)
    """
    n_right = 0
    n_wrong = 0
    
    for i in range(len(labels)):
        if truth[i] == 1 and labels[i] == 1:
                n_right += 1
        elif labels[i] == 1 and truth[i] != 1:
                n_wrong += 1
            
    return [n_right, n_wrong]


###############################################
########### Load in the MNIST data ############
###############################################
# this method comes from https://stackoverflow.com/questions/40427435/extract-images-from-idx3-ubyte-file-or-gzip-via-python
# and uses this package: https://pypi.org/project/python-mnist/
mndata = MNIST('data')

# there are 60000 items in each training object
# and 10000 in the testing
train_imgs, train_labs = mndata.load_training()
test_imgs, test_labs = mndata.load_testing()

# the imgs are now arrays of arrays of integers for each picture
# the labels are now numpy arrays
train_imgs = np.asarray(train_imgs)
test_imgs = np.asarray(test_imgs)
train_labs = np.asarray(train_labs)
test_labs = np.asarray(test_labs)

# check out the sizes
for x in [train_imgs, train_labs, test_imgs, test_labs]:
    print(x.shape)

# get the size and dimensions for each picture (n x n)
ntrain, n2 = train_imgs.shape
n = int(np.sqrt(n2)) 
ntest = test_imgs.shape[0]

# turn the list of labels into vectors
B_train = np.zeros((ntrain, 10))
B_test = np.zeros((ntest, 10))

for i in range(ntrain):
    B_train[i, :] = convert_label(train_labs[i])

for i in range(ntest):
    B_test[i, :] = convert_label(test_labs[i])
    
# plot the first few training images to see how they look
for i in range(6):
    ax = plt.subplot(2, 3, i+1)
    ax.imshow(reshape_pic(train_imgs[i]), cmap='gray')
    ax.set_title(train_labs[i])
    ax.axis('off')
plt.show()

# plot the first few testing images to see how they look
for i in range(6):
    ax = plt.subplot(2, 3, i+1)
    ax.imshow(reshape_pic(test_imgs[i]), cmap='gray')
    ax.set_title(test_labs[i])
    ax.axis('off')
plt.show()

# look at the shapes again
print(train_imgs.shape)
print(B_train.shape, '\n')
print(test_imgs.shape)
print(B_test.shape)


###############################################
####### 1. Using various AX=B Solvers #########
###############################################

# -------- (a) pseduo-inversse --------

# using the list of labels so you just get 1 loading matrix 
# (tells you the overall structure/features of all digits)
A_pinv_all = train_labs @ np.linalg.pinv(train_imgs.T)
pinv_labs_all = A_pinv_all @ test_imgs.T

# how accurate was it?
[pinv_corr_all, pinv_wrong_all] = check_accuracy_all(pinv_labs_all, test_labs)
ac_lab_pinv_all = str(pinv_corr_all/(pinv_corr_all+pinv_wrong_all)*100.) + '%'
print(ac_lab_pinv_all + ' correct')

# using the B matrix so you get 1 loading matrix for each digit
A_pinv = B_train.T @ np.linalg.pinv(train_imgs.T)
pinv_labs = A_pinv @ test_imgs.T

# how accurate was it?
[pinv_corr, pinv_wrong] = check_accuracy(pinv_labs, test_labs)
print(str(pinv_corr/(pinv_corr+pinv_wrong)*100.) + '% correct')

# check out what the weights look like - should show you 
# the important pixels!
plot_weights(A_pinv, 'Pseudo-inverse', 'pinv', save=True)
plot_weights(A_pinv, 'Pseudo-inverse', 'pinv', no_zero=True, save=True)

# some accuracy plots, etc...
plot_loadings_bar(A_pinv, 'Pseudo-inverse', 'pinv', save=True)
plot_hist_nonzero(A_pinv, 'Pseudo-inverse', 'pinv', save=True)


# -------- (b) lasso (lambda=1.0) --------

# using the list of labels so you just get 1 loading matrix 
# (tells you the overall structure/features of all digits)
lasso_all = linear_model.Lasso().fit(train_imgs, train_labs)
A_lasso_all = lasso_all.coef_
lasso_labs_1_all = A_lasso_all @ test_imgs.T

# how accurate was it?
[lasso1_corr_all, lasso1_wrong_all] = check_accuracy_all(lasso_labs_1_all, test_labs)
ac_lab_lasso_all = str(np.round(lasso1_corr_all/(lasso1_corr_all+lasso1_wrong_all)*100., 2)) + '%'
print(ac_lab_lasso_all + ' correct')

# using the B matrix
lasso = linear_model.Lasso().fit(train_imgs, B_train)
A_lasso = lasso.coef_
lasso_labs_1 = A_lasso @ test_imgs.T

# how accurate was it?
[lasso1_corr, lasso1_wrong] = check_accuracy(lasso_labs_1, test_labs)
print(str(lasso1_corr/(lasso1_corr+lasso1_wrong)*100.) + '% correct')

# check out what the weights look like - should show you 
# the important pixels!
plot_weights(A_lasso, 'Lasso (lambda=1.0)', 'lasso_1', save=True)
plot_weights(A_lasso, 'Lasso (lambda=1.0)', 'lasso_1', no_zero=True, save=True)

# some accuracy plots, etc...
plot_loadings_bar(A_lasso, 'Lasso (lambda=1.0)', 'lasso_1', save=True)
plot_hist_nonzero(A_lasso, 'Lasso (lambda=1.0)', 'lasso_1', xlim=(-2e-4, 2e-4),
                  save=True)


# -------- (c) lasso (lambda=0.5) --------

# using the list of labels so you just get 1 loading matrix 
# (tells you the overall structure/features of all digits)
lasso_05_all = linear_model.Lasso(alpha=0.5).fit(train_imgs, train_labs)
A_lasso_05_all = lasso_05_all.coef_
lasso_labs_05_all = A_lasso_05_all @ test_imgs.T

# how accurate was it?
[lasso05_corr_all, lasso05_wrong_all] = check_accuracy_all(lasso_labs_05_all, test_labs)
ac_lab_lasso_05_all = str(lasso05_corr_all/(lasso05_corr_all+lasso05_wrong_all)*100.) + '%'
print(ac_lab_lasso_05_all + ' correct')

# using the B matrix
lasso_05 = linear_model.Lasso(alpha=0.5).fit(train_imgs, B_train)
A_lasso_05 = lasso_05.coef_
lasso_labs_05 = A_lasso_05 @ test_imgs.T

# how accurate was it?
[lasso05_corr, lasso05_wrong] = check_accuracy(lasso_labs_05, test_labs)
print(str(lasso05_corr/(lasso05_corr+lasso05_wrong)*100.) + '% correct')

# check out what the weights look like - should show you 
# the important pixels!
plot_weights(A_lasso_05, 'Lasso (lambda=0.5)', 'lasso_05', save=True)
plot_weights(A_lasso_05, 'Lasso (lambda=0.5)', 'lasso_05', no_zero=True, save=True)

# some accuracy plots, etc...
plot_loadings_bar(A_lasso_05, 'Lasso (lambda=0.5)', 'lasso_05', save=True)
plot_hist_nonzero(A_lasso_05, 'Lasso (lambda=0.5)', 'lasso_05', xlim=(-2e-4, 2e-4),
                  save=True)


# -------- (d) lasso (lambda=0.1) --------

# using the list of labels so you just get 1 loading matrix 
# (tells you the overall structure/features of all digits)
lasso_01_all = linear_model.Lasso(alpha=0.1).fit(train_imgs, train_labs)
A_lasso_01_all = lasso_01_all.coef_
lasso_labs_01_all = A_lasso_01_all @ test_imgs.T

# how accurate was it?
[lasso01_corr_all, lasso01_wrong_all] = check_accuracy_all(lasso_labs_01_all, test_labs)
ac_lab_lasso_01_all = str(lasso01_corr_all/(lasso01_corr_all+lasso01_wrong_all)*100.) + '%'
print(ac_lab_lasso_01_all + ' correct')

# using the B matrix
lasso_01 = linear_model.Lasso(alpha=0.1).fit(train_imgs, B_train)
A_lasso_01 = lasso_01.coef_
lasso_labs_01 = A_lasso_01 @ test_imgs.T

# how accurate was it?
[lasso01_corr, lasso01_wrong] = check_accuracy(lasso_labs_01, test_labs)
print(str(lasso01_corr/(lasso01_corr+lasso01_wrong)*100.) + '% correct')

# check out what the weights look like - should show you 
# the important pixels!
plot_weights(A_lasso_01, 'Lasso (lambda=0.1)', 'lasso_01', save=True)
plot_weights(A_lasso_01, 'Lasso (lambda=0.1)', 'lasso_01', no_zero=True, save=True)

# some accuracy plots, etc...
plot_loadings_bar(A_lasso_01, 'Lasso (lambda=0.1)', 'lasso_01', save=True)
plot_hist_nonzero(A_lasso_01, 'Lasso (lambda=0.1)', 'lasso_01', xlim=(-2e-4, 2e-4),
                  save=True)


# -------- (d) ridge --------

# using the list of labels so you just get 1 loading matrix 
# (tells you the overall structure/features of all digits)
ridge_all = linear_model.Ridge().fit(train_imgs, train_labs)
A_ridge_all = ridge_all.coef_
ridge_labs_all = A_ridge_all @ test_imgs.T

# how accurate was it?
[ridge_corr_all, ridge_wrong_all] = check_accuracy_all(ridge_labs_all, test_labs)
ac_lab_ridge_all = str(ridge_corr_all/(ridge_corr_all+ridge_wrong_all)*100.) + '%'
print(ac_lab_ridge_all + ' correct')

# using the B matrix 
ridge = linear_model.Ridge().fit(train_imgs, B_train)
A_ridge = ridge.coef_
ridge_labs = A_ridge @ test_imgs.T

# how accurate was it?
[ridge_corr, ridge_wrong] = check_accuracy(ridge_labs, test_labs)
print(str(ridge_corr/(ridge_corr+ridge_wrong)*100.) + '% correct')

# check out what the weights look like - should show you 
# the important pixels!
plot_weights(A_ridge, 'Ridge (lambda=1.0)', 'ridge', save=True)
plot_weights(A_ridge, 'Ridge (lambda=1.0)', 'ridge', no_zero=True, save=True)

# some accuracy plots, etc...
plot_loadings_bar(A_ridge, 'Ridge (lambda=1.0)', 'ridge', save=True)
plot_hist_nonzero(A_ridge, 'Ridge (lambda=1.0)', 'ridge', save=True)


# ----- Plots for when there is one A matrix for each method -----

### look at the A matrices ###

fig, axes = plt.subplots(2, 3, figsize=(8, 6))
plt.subplots_adjust(hspace=0.3)

ax1 = axes[0, 0]
ax1.set_aspect('equal')
ax1.pcolormesh(reshape_pic(A_pinv_all), cmap='gray')
ax1.set_title('Pseudo-inverse\n{}'.format(ac_lab_pinv_all))
ax1.get_xaxis().set_ticks([])
ax1.get_yaxis().set_ticks([])

ax2 = axes[0, 1]
ax2.set_aspect('equal')
ax2.pcolormesh(reshape_pic(A_lasso_all), cmap='gray')
ax2.set_title('Lasso (lambda=1.0)\n{}'.format(ac_lab_lasso_all))
ax2.get_xaxis().set_ticks([])
ax2.get_yaxis().set_ticks([])

ax3 = axes[0, 2]
ax3.set_aspect('equal')
ax3.pcolormesh(reshape_pic(A_lasso_05_all), cmap='gray')
ax3.set_title('Lasso (lambda=0.5)\n{}'.format(ac_lab_lasso_05_all))
ax3.get_xaxis().set_ticks([])
ax3.get_yaxis().set_ticks([])

ax4 = axes[1, 0]
ax4.set_aspect('equal')
ax4.pcolormesh(reshape_pic(A_lasso_01_all), cmap='gray')
ax4.set_title('Lasso (lambda=0.1)\n{}'.format(ac_lab_lasso_01_all))
ax4.get_xaxis().set_ticks([])
ax4.get_yaxis().set_ticks([])

ax5 = axes[1, 1]
ax5.set_aspect('equal')
ax5.pcolormesh(reshape_pic(A_ridge_all), cmap='gray')
ax5.set_title('Ridge\n{}'.format(ac_lab_ridge_all))
ax5.get_xaxis().set_ticks([])
ax5.get_yaxis().set_ticks([])

axes[1, 2].axis('off')

plt.suptitle('Weights for All Digits by Method', fontsize=14)

plt.savefig(save_dir + 'weights_matrix_accuracy_all_digits_all_methods.png', dpi=300, bbox_inches='tight')

plt.show()


### bar plot of loadings ###

fig, axes = plt.subplots(2, 3, figsize=(14, 7))
plt.subplots_adjust(hspace=0.4, wspace=0.4)

ax1 = axes[0, 0]
ax1.bar(np.arange(0, 784), A_pinv_all.flatten(), color='C0')
ax1.axhline(0, linewidth=1, alpha=0.25, color='k')
ax1.set_title('Pseudo-inverse\n{}'.format(ac_lab_pinv_all))

ax2 = axes[0, 1]
ax2.bar(np.arange(0, 784), A_lasso_all.flatten(), color='C0')
ax2.axhline(0, linewidth=1, alpha=0.25, color='k')
ax2.set_title('Lasso (lambda=1.0)\n{}'.format(ac_lab_lasso_all))
        
ax3 = axes[0, 2]
ax3.bar(np.arange(0, 784), A_lasso_05_all.flatten(), color='C0')
ax3.axhline(0, linewidth=1, alpha=0.25, color='k')
ax3.set_title('Lasso (lambda=0.5)\n{}'.format(ac_lab_lasso_05_all))

ax4 = axes[1, 0]
ax4.bar(np.arange(0, 784), A_lasso_01_all.flatten(), color='C0')
ax4.axhline(0, linewidth=1, alpha=0.25, color='k')
ax4.set_title('Lasso (lambda=0.1)\n{}'.format(ac_lab_lasso_01_all))

ax5 = axes[1, 1]
ax5.bar(np.arange(0, 784), A_ridge_all.flatten(), color='C0')
ax5.axhline(0, linewidth=1, alpha=0.25, color='k')
ax5.set_title('Ridge\n{}'.format(ac_lab_ridge_all))


ylim_1 = (-0.05, 0.45)
ylim_2 = (-0.0025, 0.0055)
ylim_3 = (-0.15, 0.45)

ax1.set_ylim(ylim_1)
ax2.set_ylim(ylim_2)
ax3.set_ylim(ylim_2)
ax4.set_ylim(ylim_2)
ax5.set_ylim(ylim_3)

axes[1, 2].axis('off')

plt.suptitle('Weights for All Digits by Method', fontsize=14)

plt.savefig(save_dir + 'bar_plot_loadings_all_digits_all_methods.png', dpi=300,
            bbox_inches='tight')
    
plt.show()


###############################################
#### 2. Determine most informative pixels #####
###############################################
# NOTE: the code for #2 and #3 may overlap between sections

### plot the overall accuracy ### 

fig, ax = plt.subplots(figsize=(7, 4))

corr_counts = [pinv_corr, lasso1_corr, lasso05_corr, lasso01_corr, ridge_corr]
wrong_counts = [pinv_wrong, lasso1_wrong, lasso05_wrong, lasso01_wrong, ridge_wrong]

inds = np.arange(len(corr_counts))
width = 0.6

rects1 = ax.bar(inds, corr_counts, width, color='C0')
rects2 = ax.bar(inds, wrong_counts, width, bottom=corr_counts, color='C1')

ax.set_xticks(np.arange(0, len(corr_counts)))
ax.set_xticklabels(['Pseudo-\ninverse', 'Lasso\n(lambda=1.0)', 'Lasso\n(lambda=0.5)',
                    'Lasso\n(lambda=0.1)', 'Ridge'])

ax.set_ylabel('Number of Labels')
plt.legend([rects1[0], rects2[0]], ['Correct', 'Incorrect'], loc='upper right', ncol=2)
ax.set_title('Model Accuracy')
ax.set_ylim(0, 13000)


# label for accuracy
# https://matplotlib.org/3.1.1/gallery/lines_bars_and_markers/barchart.html#sphx-glr-gallery-lines-bars-and-markers-barchart-py
accuracy = [(corr_counts[i]/(corr_counts[i]+wrong_counts[i]))*100. for i in range(len(corr_counts))]
accuracy_labs = ['{:.2f}%'.format(x) for x in accuracy]
for i in range(len(rects2)):
    height = rects1[i].get_height() + rects2[i].get_height()
    ax.annotate(accuracy_labs[i],
                xy=(rects2[i].get_x() + rects2[i].get_width() / 2, height),
                xytext=(0, 3), textcoords='offset points',
                ha='center', va='bottom')

plt.savefig(save_dir + 'all_data_accuracy_comparison.png', dpi=300, bbox_inches='tight')

plt.show()


### look at the sparsity ###

def num_zeros(A):
    """ Print number of zero weights for each digit
    """
    num_zeros = np.zeros(10)
    for i in range(10):
        num_zeros[i] = len(A[i, :][A[i, :] == 0])
    
    print(num_zeros)
    
# from most to least sparse:
num_zeros(A_lasso)
num_zeros(A_lasso_01)
num_zeros(A_lasso_05)
num_zeros(A_ridge)
num_zeros(A_pinv)

###########################################################
## 3. Apply your most important pixels to the test data ###
###########################################################
# NOTE: the code for #2 and #3 may overlap between sections

### pick most important pixels by using some percentile as a threshold ###

pct = 90

pinv_pct = np.percentile(np.abs(A_pinv.flatten()), pct)
lasso_pct_1 = np.percentile(np.abs(A_lasso.flatten()), pct)
lasso_pct_05 = np.percentile(np.abs(A_lasso_05.flatten()), pct)
lasso_pct_01 = np.percentile(np.abs(A_lasso_01.flatten()), pct)
ridge_pct = np.percentile(np.abs(A_ridge.flatten()), pct)

new_A_pinv = np.where(A_pinv > pinv_pct, A_pinv,  0)
new_A_lasso_1 = np.where(A_lasso > lasso_pct_1, A_lasso,  0)
new_A_lasso_05 = np.where(A_lasso_05 > lasso_pct_05, A_lasso_05,  0)
new_A_lasso_01 = np.where(A_lasso_01 > lasso_pct_01, A_lasso_05,  0)
new_A_ridge = np.where(A_ridge > ridge_pct, A_ridge,  0)

new_pinv_labs = new_A_pinv @ test_imgs.T
new_lasso_labs = new_A_lasso_1 @ test_imgs.T
new_lasso_05_labs = new_A_lasso_05 @ test_imgs.T
new_lasso_01_labs = new_A_lasso_01 @ test_imgs.T
new_ridge_labs = new_A_ridge @ test_imgs.T

### accuracy with that subset ###

lab_list = [new_pinv_labs, new_lasso_labs, new_lasso_05_labs,
            new_lasso_01_labs, new_ridge_labs]

new_corr_counts = [[]]*len(lab_list)
new_wrong_counts = [[]]*len(lab_list)

for i in range(len(lab_list)):
    new_corr_counts[i], new_wrong_counts[i] = check_accuracy(lab_list[i], test_labs)
    
### plot it ###

pct=90
fig, ax = plt.subplots(figsize=(7, 4))

inds = np.arange(len(corr_counts))
width = 0.6

rects1 = ax.bar(inds, new_corr_counts, width, color='C0')
rects2 = ax.bar(inds, new_wrong_counts, width, bottom=new_corr_counts, color='C1')

ax.set_xticks(np.arange(0, len(new_corr_counts)))
ax.set_xticklabels(['Pseudo-\ninverse', 'Lasso\n(lambda=1.0)', 'Lasso\n(lambda=0.5)',
                    'Lasso\n(lambda=0.1)', 'Ridge'])

ax.set_ylabel('Number of Labels')
plt.legend([rects1[0], rects2[0]], ['Correct', 'Incorrect'], loc='upper right', ncol=2)
ax.set_title('Model Accuracy (with weights >= {}th percentile)'.format(pct))
ax.set_ylim(0, 13000)


# label for accuracy
# https://matplotlib.org/3.1.1/gallery/lines_bars_and_markers/barchart.html#sphx-glr-gallery-lines-bars-and-markers-barchart-py
accuracy = [(new_corr_counts[i]/(new_corr_counts[i]+new_wrong_counts[i]))*100. for i in range(len(new_corr_counts))]
accuracy_labs = ['{:.2f}%'.format(x) for x in accuracy]
for i in range(len(rects2)):
    height = rects1[i].get_height() + rects2[i].get_height()
    ax.annotate(accuracy_labs[i],
                xy=(rects2[i].get_x() + rects2[i].get_width() / 2, height),
                xytext=(0, 3), textcoords='offset points',
                ha='center', va='bottom')

plt.savefig(save_dir + 'geq_{}th_pct_accuracy_comparison.png'.format(pct), dpi=300, bbox_inches='tight')

plt.show()

### from most to least sparse with the cuttoff: ###

print('{}th percentile'.format(pct))
num_zeros(new_A_lasso_1)
num_zeros(new_A_lasso_01)
num_zeros(new_A_lasso_05)
num_zeros(new_A_ridge)
num_zeros(new_A_pinv)

### plot these "most important" pixels ###

plot_weights(new_A_pinv, 'Pseudo-inverse', 'pinv_geq_90th', no_zero=True, save=True)
plot_weights(new_A_lasso, 'Lasso (lambda=1.0)', 'lasso_1_geq_90th', no_zero=True, save=True)
plot_weights(new_A_lasso_05, 'Lasso (lambda=0.5)', 'lasso_05_geq_90th', no_zero=True, save=True)
plot_weights(new_A_lasso_01, 'Lasso (lambda=0.1)', 'lasso_01_geq_90th', no_zero=True, save=True)
plot_weights(new_A_ridge, 'Ridge (lambda=1.0)', 'ridge_geq_90th', no_zero=True, save=True)


### test the cutoff with Lasso (lambda=1.0) ###

xspace = np.arange(0, 100.1, .1)
plt.plot(xspace, np.percentile(np.abs(A_lasso).flatten(), xspace))
plt.title('Value of the Xth Percentile of Entries\nin the Lasso (lambda=1.0) matrix')
plt.xlabel('percentile')
plt.ylabel('loading value')

plt.savefig(save_dir + 'lasso_loading_percentiles-see_uptick_at_90.png',
            dpi=300, bbox_inches='tight')
plt.show()

### pick most important pixels by using some percentile as a threshold ###

lasso90 = np.percentile(np.abs(A_lasso.flatten()), 90)
lasso95 = np.percentile(np.abs(A_lasso.flatten()), 95)
lasso975 = np.percentile(np.abs(A_lasso.flatten()), 97.5)
lasso99 = np.percentile(np.abs(A_lasso.flatten()), 99)

A_lasso_90 = np.where(np.abs(A_lasso) >= lasso90, A_lasso,  0)
A_lasso_95 = np.where(np.abs(A_lasso) >= lasso95, A_lasso,  0)
A_lasso_975 = np.where(np.abs(A_lasso) >= lasso975, A_lasso,  0)
A_lasso_99 = np.where(np.abs(A_lasso) >= lasso99, A_lasso,  0)

lasso_labs_90 = A_lasso_90 @ test_imgs.T
lasso_labs_95 = A_lasso_95 @ test_imgs.T
lasso_labs_975 = A_lasso_975 @ test_imgs.T
lasso_labs_99 = A_lasso_99 @ test_imgs.T


lasso_lab_list = [lasso_labs_1, lasso_labs_90, lasso_labs_95, lasso_labs_975,
                  lasso_labs_99]

lasso_corr_counts = [[]]*len(lasso_lab_list)
lasso_wrong_counts = [[]]*len(lasso_lab_list)

for i in range(len(lasso_lab_list)):
    lasso_corr_counts[i], lasso_wrong_counts[i] = check_accuracy(lasso_lab_list[i], test_labs)

### plot accuracy by pctl ###

fig, ax = plt.subplots(figsize=(7, 4))

inds = np.arange(len(lasso_corr_counts))
width = 0.6

rects1 = ax.bar(inds, lasso_corr_counts, width, color='C0')
rects2 = ax.bar(inds, lasso_wrong_counts, width, bottom=lasso_corr_counts, color='C1')

ax.set_xticks(np.arange(0, len(lasso_corr_counts)))
ax.set_xticklabels(['All', '$\geq$90th pctl.', '$\geq$95th pctl.',
                    '$\geq$97.5th pctl.', '$\geq$99th pctl.'])

ax.set_ylabel('Number of Labels')
plt.legend([rects1[0], rects2[0]], ['Correct', 'Incorrect'], loc='upper right', ncol=2)
ax.set_title('Model Accuracy (Lasso with lambda=1.0)')
ax.set_ylim(0, 13000)

# label for accuracy
# https://matplotlib.org/3.1.1/gallery/lines_bars_and_markers/barchart.html#sphx-glr-gallery-lines-bars-and-markers-barchart-py
accuracy = [(lasso_corr_counts[i]/(lasso_corr_counts[i]+lasso_wrong_counts[i]))*100. for i in range(len(lasso_corr_counts))]
accuracy_labs = ['{:.2f}%'.format(x) for x in accuracy]
for i in range(len(rects2)):
    height = rects1[i].get_height() + rects2[i].get_height()
    ax.annotate(accuracy_labs[i],
                xy=(rects2[i].get_x() + rects2[i].get_width() / 2, height),
                xytext=(0, 3), textcoords='offset points',
                ha='center', va='bottom')

plt.savefig(save_dir + 'lasso_pctl_accuracy_comparison.png', dpi=300, bbox_inches='tight')

plt.show()


###########################################################
#### 4. Redo the analysis with each digit individually ####
###########################################################

# -------- (a) all pixels, lasso (lambda=1.0) --------

### change the B vectors to be specific for each digit ###

B_0 = np.where(train_labs==0, 1, 0)
B_1 = np.where(train_labs==1, 1, 0)
B_2 = np.where(train_labs==2, 1, 0)
B_3 = np.where(train_labs==3, 1, 0)
B_4 = np.where(train_labs==4, 1, 0)
B_5 = np.where(train_labs==5, 1, 0)
B_6 = np.where(train_labs==6, 1, 0)
B_7 = np.where(train_labs==7, 1, 0)
B_8 = np.where(train_labs==8, 1, 0)
B_9 = np.where(train_labs==9, 1, 0)

B_list = [B_1, B_2, B_3, B_4, B_5, B_6, B_7, B_8,
          B_9, B_0]

### generate the loadings matrix for all digits (using all pixels) ###

A_lasso_digits = [[]]*10
lasso_digit_labels = [[]]*10

for i in range(10):
    lasso_digit = linear_model.Lasso().fit(train_imgs, B_list[i])
    A_lasso_digits[i] = lasso_digit.coef_
    lasso_digit_labels[i] = np.sign(A_lasso_digits[i] @ test_imgs.T)

# quick look at the A matrices (but they're exactly the same as before!)
for x in A_lasso_digits:
    plt.imshow(reshape_pic(x), cmap='gray')
    plt.show()

### plot accuracy for each digit using all pixels ###

test_labs_0 = np.where(test_labs == 0, 1, 0)
test_labs_1 = np.where(test_labs == 1, 1, 0)
test_labs_2 = np.where(test_labs == 2, 1, 0)
test_labs_3 = np.where(test_labs == 3, 1, 0)
test_labs_4 = np.where(test_labs == 4, 1, 0)
test_labs_5 = np.where(test_labs == 5, 1, 0)
test_labs_6 = np.where(test_labs == 6, 1, 0)
test_labs_7 = np.where(test_labs == 7, 1, 0)
test_labs_8 = np.where(test_labs == 8, 1, 0)
test_labs_9 = np.where(test_labs == 9, 1, 0)

test_lab_list = [test_labs_1, test_labs_2, test_labs_3, test_labs_4,
                 test_labs_5, test_labs_6, test_labs_7, test_labs_8,
                 test_labs_9, test_labs_0]

### plot the first few to see how they look ###

for j in range(10):
    if j == 9:
        num = 0
    else:
        num = j+1
        
    # indices where the picture is of digit #num  
    digit_inds = np.where(test_lab_list[j] == 1)[0]
    not_digit_inds = np.where(test_lab_list[j] != 1)[0]

    # indices where there is a false positive of digit #num
    not_digit_inds = np.where(test_lab_list[j] != 1)[0]
    false_pos_inds = []
    for n in not_digit_inds:
        if lasso_digit_labels[j][n] == 1:
            false_pos_inds.append(n)
    
    for i in range(3):
        # randomly select a digit out of the list
        ind = random.choice(digit_inds)
        
        ax = plt.subplot(2, 3, i+1)
        ax.imshow(reshape_pic(test_imgs[ind]), cmap='gray')
        if lasso_digit_labels[j][ind] == 1:
            ax.set_title('Identified', fontsize=11)
        else:
            ax.set_title('Missed', fontsize=11)
        ax.axis('off')

    for k in range(3, 6):
        # randomly select a digit out of the list
        ind = random.choice(false_pos_inds)
        
        ax = plt.subplot(2, 3, k+1)
        ax.imshow(reshape_pic(test_imgs[ind]), cmap='gray')
        ax.set_title('Identified as {}'.format(num), fontsize=11)
        ax.axis('off')        
    
    plt.suptitle('Random Sample of Lasso (lambda=1.0) Results for {}'.format(num),
                 fontsize=14)
    name = 'ALL_PIXELS_ex_of_correct_and_false_pos_for_{}'.format(num)
    plt.savefig(save_dir + name, dpi=300, bbox_inches='tight')
    
    plt.show()


### calculate accuracies ###

corr_list_digit = [[]]*10
wrong_list_digit = [[]]*10
accuracy_digit_list = [[]]*10
false_pos_list = [[]]*10
true_pos_list = [[]]*10

for i in range(10):
    if i == 9:
        n = 0
    else:
        n = i+1
    corr_list_digit[i], wrong_list_digit[i] = check_accuracy_digit(lasso_digit_labels[i],
                                                                   test_lab_list[i])
    
    # how often the digit was identified correctly and the not-digits were
    # identified correctly
    accuracy_digit_list[i] = corr_list_digit[i]/(corr_list_digit[i] + wrong_list_digit[i])*100.
    
    fp = 0
    tp = 0
    for j in range(ntest):
        if lasso_digit_labels[i][j] == 1 and test_lab_list[i][j] != 1:
            fp += 1
        elif lasso_digit_labels[i][j] == 1 and test_lab_list[i][j] == 1:
            tp += 1
    
    # how often the digit was identified when it wasn't the true label
    # and how often the digit was identified correctly
    false_pos_list[i] = fp/ntest*100.
    true_pos_list[i] = tp/np.sum(test_lab_list[i])*100.
    
    print('{n}: {a}% correct'.format(n=n, a=accuracy_digit_list[i]))
    print('{n}: {f}% false positives'.format(n=n, f=false_pos_list[i]))
    print('{n}: {t}% true positives'.format(n=n, t=true_pos_list[i]))
    

### bar chart for accuracy by digit ###

fig, ax = plt.subplots(figsize=(8, 4))

inds = np.arange(10)
width = 0.2

rects_corr = ax.bar(inds, corr_list_digit, width, color='C0')
rects_wrong = ax.bar(inds, wrong_list_digit, width, bottom=corr_list_digit, color='C1')

ax.set_xticks(np.arange(0.25, 10.25))
ax.set_xticklabels([*np.arange(1, 10), 0])
ax.set_ylabel('Number of Labels')
ax.set_ylim(0, 13000)

ax2 = ax.twinx()
ax2.set_ylabel('Incidence (%)')
ax2.set_ylim(0, 130)
rects_tp = ax2.bar(inds+0.5, true_pos_list, width, color='C3')
rects_fp = ax2.bar(inds+0.25, false_pos_list, width, color='C2')

plt.legend([rects_corr[0], rects_wrong[0], rects_tp[0], rects_fp[0]],
           ['Correct', 'Incorrect', 'True Positive', 'False Positive'],
           loc='upper right', ncol=2)
ax.set_title('Model Accuracy by Digit (Lasso with lambda=1.0)\n(All Pixels)')

plt.savefig(save_dir + 'DIGIT_ALL_PIXELS_lasso_accuracy_comparison.png', dpi=300, bbox_inches='tight')

plt.show()

### make a table of the percentages ###
dig_list = [*np.arange(1, 10), 0]
acc_labs_dig = ['{:.1f}%'.format(x) for x in accuracy_digit_list]
fp_labs = ['{:.1f}%'.format(x) for x in false_pos_list]
tp_labs = ['{:.1f}%'.format(x) for x in true_pos_list]

cols = [[]]*10
for i in range(10):
    cols[i] = [dig_list[i], acc_labs_dig[i], tp_labs[i], fp_labs[i]]

print('All Pixels:')
print(tabulate(cols, ['Digit', 'Accuracy', 'True Positive Rate', 'False Positive Rate'],
               tablefmt='fancy_grid'))


# -------- (b) pick the most important pixels for each digit --------

### test the cutoff ###

xspace = np.arange(0, 100.1, .1)

for i in range(10):
    if i == 9:
        num = 0
    else:
        num = i + 1
    plt.plot(xspace, np.percentile(np.abs(A_lasso_digits[i]).flatten(), xspace),
             label=num)

plt.legend()    
plt.xlim((85, 100.5))
plt.title('Value of the Xth Percentile of Entries\nin the Lasso (lambda=1.0) matrix')
plt.xlabel('percentile')
plt.ylabel('loading value')

plt.savefig(save_dir + 'DIGIT_lasso_loading_percentiles-see_uptick_at_91.png',
            dpi=300, bbox_inches='tight')
plt.show()

### pick the most important pixels by using some percentile as a threshold ###

# pick most important pixels by using some percentile as a treshold

pct_list = [0, 91, 95, 97.5, 99]
digit_pcts = [[]]*len(pct_list)
new_A_digits = [[]]*len(pct_list)
new_digit_labs = [[]]*len(pct_list)

# 0
digit_pcts[0] = [np.percentile(np.abs(x.flatten()), pct_list[0]) 
                 for x in A_lasso_digits]
new_A_digits[0] = [np.where(np.abs(A_lasso_digits[i]) >= digit_pcts[0][i],
                            A_lasso_digits[i], 0)
                   for i in range(10)]
new_digit_labs[0] = [np.sign(new_A_digits[0][i] @ test_imgs.T)
                     for i in range(10)]

# 91
digit_pcts[1] = [np.percentile(np.abs(x.flatten()), pct_list[1]) 
                 for x in A_lasso_digits]
new_A_digits[1] = [np.where(np.abs(A_lasso_digits[i]) >= digit_pcts[1][i],
                            A_lasso_digits[i], 0)
                   for i in range(10)]
new_digit_labs[1] = [np.sign(new_A_digits[1][i] @ test_imgs.T)
                     for i in range(10)]

# 95
digit_pcts[2] = [np.percentile(np.abs(x.flatten()), pct_list[2]) 
                 for x in A_lasso_digits]
new_A_digits[2] = [np.where(np.abs(A_lasso_digits[i]) >= digit_pcts[2][i],
                            A_lasso_digits[i], 0)
                   for i in range(10)]
new_digit_labs[2] = [np.sign(new_A_digits[2][i] @ test_imgs.T)
                     for i in range(10)]

# 97.5
digit_pcts[3] = [np.percentile(np.abs(x.flatten()), pct_list[3]) 
                 for x in A_lasso_digits]
new_A_digits[3] = [np.where(np.abs(A_lasso_digits[i]) >= digit_pcts[3][i],
                            A_lasso_digits[i], 0)
                   for i in range(10)]
new_digit_labs[3] = [np.sign(new_A_digits[3][i] @ test_imgs.T)
                     for i in range(10)]

# 99
digit_pcts[4] = [np.percentile(np.abs(x.flatten()), pct_list[4]) 
                 for x in A_lasso_digits]
new_A_digits[4] = [np.where(np.abs(A_lasso_digits[i]) >= digit_pcts[4][i],
                            A_lasso_digits[i], 0)
                   for i in range(10)]
new_digit_labs[4] = [np.sign(new_A_digits[4][i] @ test_imgs.T)
                     for i in range(10)]


### get the % accuracy by choice of important pixels ###

corr_list_digit_pcts = [[]]*len(pct_list)
wrong_list_digit_pcts = [[]]*len(pct_list)
accuracy_digit_list_pcts = [[]]*len(pct_list)
false_pos_list_pcts = [[]]*len(pct_list)
true_pos_list_pcts = [[]]*len(pct_list)

for j in range(len(pct_list)):
    corr_list_dig = [[]]*10
    wrong_list_dig = [[]]*10
    accuracy_digit_list = [[]]*10
    false_pos_dig = [[]]*10
    true_pos_dig = [[]]*10

    for i in range(10):
        corr_list_dig[i], wrong_list_dig[i] = check_accuracy_digit(new_digit_labs[j][i],
                                                                   test_lab_list[i])

        # how often the digit was identified correctly and the not-digits were
        # identified correctly
        accuracy_digit_list[i] = corr_list_dig[i]/(corr_list_dig[i] + wrong_list_dig[i])*100.
        
        fp = 0
        tp = 0
        for k in range(ntest):
            if new_digit_labs[j][i][k] == 1 and test_lab_list[i][k] != 1:
                fp += 1
            elif new_digit_labs[j][i][k] == 1 and test_lab_list[i][k] == 1:
                tp += 1
    
        # how often the digit was identified when it wasn't the true label
        # and how often the digit was identified correctly
        false_pos_dig[i] = fp/ntest*100.
        true_pos_dig[i] = tp/np.sum(test_lab_list[i])*100.
    
    corr_list_digit_pcts[j] = corr_list_dig
    wrong_list_digit_pcts[j] = wrong_list_dig
    accuracy_digit_list_pcts[j] = accuracy_digit_list
    false_pos_list_pcts[j] = false_pos_dig
    true_pos_list_pcts[j] = true_pos_dig

### plot the % accuracy ###

fig, axes = plt.subplots(2, 3, figsize=(19, 7))
plt.subplots_adjust(hspace=0.5, wspace=0.3)

inds = np.arange(10)
width = 0.8

for r in range(2):
    for c in range(3):
        ax = axes[r, c]
        
        if r == 0:
            i = c
        else:
            i = c+3
        
        if i != 5:  
            rects_corr = ax.bar(inds, corr_list_digit_pcts[i], width, color='C0')
            rects_wrong = ax.bar(inds, wrong_list_digit_pcts[i], width, bottom=corr_list_digit_pcts[i], color='C1')

            ax.set_xticks(np.arange(0, 10))
            ax.set_xticklabels([*np.arange(1, 10), 0], fontsize=16)
            ax.set_ylabel('# Labels', fontsize=14)
            ax.set_ylim(0, 12000)
            if i == 0:
                ax.set_title('All Pixels', fontsize=18)
            elif i == 1:
                ax.set_title('$\geq${}st pctl.'.format(pct_list[i]), fontsize=18)
            else:
                ax.set_title('$\geq${}th pctl.'.format(pct_list[i]), fontsize=18)

            # label for accuracy
            # https://matplotlib.org/3.1.1/gallery/lines_bars_and_markers/barchart.html#sphx-glr-gallery-lines-bars-and-markers-barchart-py
            acc_labs_pct = ['{:.1f}%'.format(x) for x in accuracy_digit_list_pcts[i]]
            for j in range(len(rects_corr)):
                height = rects_corr[j].get_height() + rects_wrong[j].get_height()
                ax.annotate(acc_labs_pct[j],
                            xy=(rects_corr[j].get_x() + rects_wrong[j].get_width() / 2, height),
                            xytext=(0, 3), textcoords='offset points',
                            ha='center', va='bottom')
            
        else:
        # proxy legend
            ax.axis('off')
            corr_patch = mpatches.Patch(color='C0', label='Correct')
            wrong_patch = mpatches.Patch(color='C1', label='Incorrect')
            ax.legend(handles=[corr_patch, wrong_patch], fontsize=18,
                      loc='center')

plt.suptitle('Model Accuracy by Digit (Lasso with lambda=1.0)', fontsize=20)

plt.savefig(save_dir + 'DIGIT_PCTS_lasso_accuracy_comparison.png', dpi=300, bbox_inches='tight')

plt.show()

### make tables of percentages ###

# 91st
dig_list = [*np.arange(1, 10), 0]
acc_labs_dig = ['{:.1f}%'.format(x) for x in accuracy_digit_list_pcts[1]]
fp_labs = ['{:.1f}%'.format(x) for x in false_pos_list_pcts[1]]
tp_labs = ['{:.1f}%'.format(x) for x in true_pos_list_pcts[1]]

cols = [[]]*10
for i in range(10):
    cols[i] = [dig_list[i], acc_labs_dig[i], tp_labs[i], fp_labs[i]]

print('>= 91st percentile')
print(tabulate(cols, ['Digit', 'Accuracy', 'True Positive Rate', 'False Positive Rate'],
               tablefmt='fancy_grid'))

# 95th
dig_list = [*np.arange(1, 10), 0]
acc_labs_dig = ['{:.1f}%'.format(x) for x in accuracy_digit_list_pcts[2]]
fp_labs = ['{:.1f}%'.format(x) for x in false_pos_list_pcts[2]]
tp_labs = ['{:.1f}%'.format(x) for x in true_pos_list_pcts[2]]

cols = [[]]*10
for i in range(10):
    cols[i] = [dig_list[i], acc_labs_dig[i], tp_labs[i], fp_labs[i]]

print('>= 95th percentile')
print(tabulate(cols, ['Digit', 'Accuracy', 'True Positive Rate', 'False Positive Rate'],
               tablefmt='fancy_grid'))

# 97.5th
dig_list = [*np.arange(1, 10), 0]
acc_labs_dig = ['{:.1f}%'.format(x) for x in accuracy_digit_list_pcts[3]]
fp_labs = ['{:.1f}%'.format(x) for x in false_pos_list_pcts[3]]
tp_labs = ['{:.1f}%'.format(x) for x in true_pos_list_pcts[3]]

cols = [[]]*10
for i in range(10):
    cols[i] = [dig_list[i], acc_labs_dig[i], tp_labs[i], fp_labs[i]]

print('>= 97.5th percentile')
print(tabulate(cols, ['Digit', 'Accuracy', 'True Positive Rate', 'False Positive Rate'],
               tablefmt='fancy_grid'))

# 99th 
dig_list = [*np.arange(1, 10), 0]
acc_labs_dig = ['{:.1f}%'.format(x) for x in accuracy_digit_list_pcts[4]]
fp_labs = ['{:.1f}%'.format(x) for x in false_pos_list_pcts[4]]
tp_labs = ['{:.1f}%'.format(x) for x in true_pos_list_pcts[4]]

cols = [[]]*10
for i in range(10):
    cols[i] = [dig_list[i], acc_labs_dig[i], tp_labs[i], fp_labs[i]]

print('>= 99th percentile')
print(tabulate(cols, ['Digit', 'Accuracy', 'True Positive Rate', 'False Positive Rate'],
               tablefmt='fancy_grid'))

\end{lstlisting}


\end{document}